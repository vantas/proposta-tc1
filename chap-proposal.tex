\chapter{\label{chap:proposal}Proposta do Trabalho}

% TODO - vamos fazer o simulador ou usar um pronto? DEFINIR
{\color{red}A proposta deste trabalho é comparar diferentes estratégias de
  controle de elevadores em alguns cenários e avaliar, dentre as opções
  possíveis, quais combinações resultam em um melhor desempenho no transporte de
  passageiros. A comparação se dará analisando os resultados de simulações de
  diferentes algoritmos de controle. Para tanto, serão implementados no mínimo 2
  algoritmos de IA, além de um algoritmo trivial para comparação, para o sistema
 de controle de elevadores e um simulador de elevadores. O usuário do simulador poderá:

\begin{itemize}
  \item Selecionar cenários;
  \item Selecionar um algoritmo para controle dos elevadores ou implementar o seu próprio;
  \item Comparar o desempenho entre diferentes algoritmos em um mesmo cenário;
  \item Comparar o desempenho de um algoritmo em múltiplos cenários.
\end{itemize}

Com base nos resultados estatísticos obtidos através das simulações serão realizadas análises quantitativas e qualitativas e, se possível, otimizações nas implementações dos algoritmos. Assim, objetiva-se encontrar quais são as melhores estratégias para cada cenário proposto.}

MODELAGEM

AGENTES

FUNÇÃO-OBJETIVO