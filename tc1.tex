\documentclass[portuguese,oneside]{tcc}

\usepackage{graphicx}
\usepackage{multirow}
\usepackage{nicefrac}
\usepackage{algpseudocode}
%\usepackage{algorithmic}
\usepackage{calc}
\usepackage{enumitem}
\usepackage{fixltx2e}

\newcommand{\argmin}[1]{\underset{#1}{\operatorname{arg}\,\operatorname{min}}\;}
\newcommand{\argmax}[1]{\underset{#1}{\operatorname{arg}\,\operatorname{max}}\;}

\author{Guilherme de Mello Mattos Taschetto e Pedro Pillon Vanzella}

\title{Inteligência Artificial Aplicada a Elevadores}
      {Artificial Intelligence Applied to Elevators}

\tipotrabalho{\tci}
\curso{\cc}
\orientador{João Batista de Oliveira}

\begin{document}

\begin{resumo}{elevadores, inteligência artificial, simulação, sistemas multiagentes, aprendizado de máquina}

Elevadores são um meio de transporte utilizado por milhões de pessoas no mundo
inteiro. Com este trabalho, pretende-se propor soluções de Inteligência
Artificial para melhorar a eficiência dos mesmos, reduzindo o tempo que
passageiros despendem em função destes. Através da modelagem e implementação de
um simulador de elevadores e de algoritmos de Inteligência Artificial, espera-se
realizar testes de diferentes técnicas e soluções propostas na literatura, a fim
de propor quais estratégias são mais adequadas para cada cenário.

\end{resumo}

\begin{abstract}{elevators, artificial intelligence, simulation, multiagent systems, machine learning}

Elevators are a mode of transportation used by milions of people around the
world. In this project, we will propose Artificial Intelligence solutions in
order to increase their efficiency, reducing their users' wait times. Through
the process of modeling and implementation of an elevator simulator and
Artificial Intelligence algorithms, we hope to test out the different techniques
and solutions proposed in the literature, in order to propose which strategies
best fit each scenario.

\end{abstract}

\tableofcontents

\chapter{\label{chap:intro}Introdução}

\sigla{EGCS}{Elevator Group Control System}

Em 2014, 54\% da população mundial vivia em áreas urbanas, de acordo com a
Organização das Nações Unidas~\cite{UN14}. A expectativa é que esta proporção
aumente para 66\% até o ano 2050. Em números absolutos isto representa um
acréscimo de 2,5 bilhões de pessoas à população urbana mundial nos próximos 35
anos. Uma das consequências da alta densidade populacional em regiões
geográficas limitadas é o crescimento do modelo de verticalização na construção
civil. Neste cenário, onde prédios de diversos andares se tornam presença no
cotidiano da maioria da população, os elevadores\footnote{Dispositivos de
transporte vertical que movimenta pessoas ou cargas entre andares ou níveis de
um prédio ou estrutura.} passam a um papel de destaque.

Uma pesquisa realizada pela IBM (figura \ref{fig:timecost}) no ano de 2010 em 16
cidades norte-americanas constatou que, durante 12 meses, o tempo estimado no
qual trabalhadores de escritórios\footnote{Em uma força de trabalho total de 51
milhões de trabalhadores, dos quais 12,7 milhões são usuários de elevadores
diariamente~\cite{IBM10}.} aguardaram por elevadores foi de 92
anos~\cite{IBM10}. Em uma economia onde o salário horário médio de um
trabalhador é de US\$ 24,99, o tempo de espera por elevadores representa custos
de mais de US\$ 20 bilhões em média por ano~\cite{BLS15}.

Além do impacto econômico existe o impacto psicológico. Trabalhadores em centros
metropolitanos empreendem uma parcela significativa da sua rotina no
deslocamento entre residência e local de trabalho e no caminho inverso ao final
do dia. Além de gastar uma quantidade significativa de tempo no trânsito das
ruas, em carros, ônibus, bicicletas e metrôs, o tempo compreendido entre
aguardar o elevador e desembarcar no andar desejado está longe de ser
desprezível. De acordo com reportagem da revista Time, o \textit{everyday
commute}, ou \textit{translado casa-trabalho e trabalho-casa} em uma tradução
livre, pode causar uma série de efeitos físicos, como aumento nos níveis de
açúcar, colesterol e dores nas costas, e psicológicos, como aumento na ansiedade
e depressão \cite{Kylstra14}.

\begin{figure}[htb!]
\centering\includegraphics{img/time-cost.eps}
\caption{\label{fig:timecost}Tempo de espera acumulado (em anos) por elevadores
durante 12 meses em 16 cidades norte-americanas. Fonte:~\cite{IBM10}}
\end{figure}

Neste contexto global, a indústria de elevadores possui alguns desafios:
primeiro, lidar com a pressão para a redução de custos na construção civil,
construindo sistemas de grupos de elevadores mais baratos e eficientes, com
melhorias no desempenho de transporte; segundo, competir no mercado oferecendo
serviços novos, personalizados e com garantia de qualidade, visando revolucionar
a maneira com que elevadores interagem e servem
passageiros~\cite{KOEHLEROTTIGER02}. Já sob o ponto de vista dos passageiros,
estes esperam que suas chamadas sejam atendidas imediatamente e que sejam
levados ao seu destino o mais rápido possível. Portanto, melhorias no desempenho
destes sistemas traduzem-se em alto valor tanto para os fabricantes quanto para
os usuários de elevadores.

Existem diversas abordagens que os fabricantes de elevadores podem usar para
tornar o sistema mais eficiente. Por exemplo, projetar o sistema com um número
maior de elevadores ou optar por elevadores com maior capacidade de carga.
Entretanto, este tipo de alteração não é sempre realizável em função de
limitações na estrutura do prédio ou inviabilidade financeira. Uma solução de
mais fácil aplicação é otimizar o sistema de controle dos elevadores.

A função deste sistema é resolver um problema de otimização: atribuir elevadores
para atender chamadas feitas pelos passageiros minimizando alguma métrica - mais
adiante serão apresentadas as métricas de desempenho utilizadas. Entretanto,
este problema encontra-se no conjunto de problemas NP-difícil (ou NP-hard, ou
NP-completo)~\cite{SeKo99}. Portanto, uma solução ótima, computável em tempo
polinomial, ainda não é conhecida para este problema.

Desde meados dos anos 1980, a indústria de elevadores vem estudando e
implementando estratégias para encontrar soluções sub-ótimas para o problema de
otimização. Diversas técnicas de Inteligência Artificial foram adotadas, como
redes neurais, algoritmos genéticos, lógicas \textit{fuzzy} e, mais
recentemente, sistemas multi-agentes, planejamento e aprendizado de
máquina~\cite{KOEHLEROTTIGER02}. Porém, melhorias significativas apenas são
encontradas nos modelos de ponta, que adotam novos paradigmas de utilização e
interface com usuários, e acabam ausentes dos modelos mais simples e
legados, i.e.,~que já estão instalados nos prédios.

Ao analisar a tecnologia atual, percebe-se que houve pouca evolução nos
elevadores desde sua concepção~-~sobretudo nos sistemas mais simples e mais
baratos. De fato, o mecanismo que os faz mover evoluiu drasticamente, fato
evidenciado ao comparar máquinas de tração manual com máquinas elétricas
modernas. No entanto, essas evoluções são de difícil percepção aos usuários de
elevadores. O sistema de controle dos elevadores ainda é simples, utilizando-se
muito pouco de técnicas de Inteligência Artificial para alterar seu
comportamento em tempo de execução, apesar dos esforços da indústria ao longo
das últimas décadas.

Acredita-se que tais técnicas permitam trazer uma evolução grande para os
sistemas de controle de grupos de elevadores, assim como os motores elétricos
foram para a tração. Por isto, o objetivo desde trabalho é comparar, através de
simulações, diferentes estratégias de controle de elevadores utilizando
Inteligência Artificial em alguns cenários. Assim, espera-se ser possível
avaliar, dentre as opções possíveis, quais combinações resultam em um melhor
desempenho no transporte de passageiros para cada cenário.

\section{\label{section:motivation}Motivação Prática}

O uso de elevadores é presente na vida de grande parte dos habitantes de grandes
metrópoles. Uma parcela significativa do dia-a-dia desta população é passada em
grandes caixas metálicas, ou esperando pelas mesmas. A possibilidade de aplicar
conhecimentos em Inteligência Artificial para melhorar o desempenho deste meio
de transporte e, por conseqüência, contribuir com um aumento na qualidade de
vida de seus passageiros é um grande atrativo para o estudo deste assunto.

Além disso, a Inteligência Artifical é uma área de conhecimento de grande
interesse dos autores, junto dos problemas relacionados a simulações, às
ferramentas de programação e à engenharia de software. O problema de otimização
na alocação de elevadores apresenta um campo de pesquisa único para a aplicação
destas ferramentas em busca da prática da Inteligência Artificial.

\include{chap-problem}
\include{chap-proposal}
\chapter{\label{chap:ai}Algoritmos de Inteligência Artificial para Elevadores}

Neste capítulo vamos apresentar:

\begin{itemize}
\item Os detalhes dos algoritmos selecionados e justificar a suas escolhas perante o problema;
\item Introduzir o modelo de simulação do sistema~-~ou seja, a modelagem
prédio/andares/elevadores (e não a modelagem do simulador propriamente dito);
\item Detalhes de cada algoritmo e descrever como esperamos que seja o resultado de seu uso junto ao sistema.
\end{itemize}

A busca pela solução do problema de atribuir elevadores para atender chamadas
feitas pelos passageiros, minimizando alguma métrica, é apresentada pela
literatura pesquisada na forma de algoritmos com complexidades distintas. Tais
complexidades vão desde algoritmos triviais, que sequer podem ser classificados
como algoritmos de Inteligência Artificial mas são interessantes para fins de
comparações, até soluções mais complexas, onde são necessários menores níveis de
abstração - ou seja, mais dados sobre o sistema são necessários para se chegar
à uma solução.

Em um hipotético cenário ideal, ter-se-ia todos os dados de cada
passageiro~-~\textit{i.e.} cada pessoa que chegasse ao elevador informaria de
antemão para qual andar desejaria ir. No entanto, isto não é realista no
contexto dos sistemas de elevadoras instalados atualmente, onde cada pessoa
apenas informa se deseja subir ou descer antes de sequer entrar em um elevador
e, conforme explicado na \ref{chap:problem}, é muito difícil saber com precisão
quantas pessoas estão esperando no corredor de cada andar. Portanto, os
algoritmos aqui descritos tentam fazer inferências a respeito de dados que não
possuem, quando relevante\footnote{\textit{e.g.} Podemos estimar a lotação do
elevador com base no peso reportado pela balança interna do elevador, que já se
encontra nele por motivos de segurança.}, ou tentam tomar decisões ignorando os
dados que não estão disponíveis.

Neste estudo serão apresentadas duas abordagens para encontrar uma solução para
o problema referido: a \textit{minimização de uma função de custo} e
\textit{planning}.

\section{\label{sec:ai:minimize-cost-function}Minimização da Função de Custo}

A primeira abordagem apresentada é simples: define-se uma função de custo,
inerente a cada elevador, que descreve matematicamente quão vantajoso é atender
um chamado, comparado a não atendê-lo. A decisão de qual elevador é escolhido
para atender o chamado é feita com base única e exclusivamente em qual deles
terá o menor resultado da função de custo. O algoritmo \ref{alg:base} fornece um
esqueleto utilizado para avaliar diferentes funções de custo.

\begin{algorithm}[htb]
\begin{center}
\begin{algorithmic}[1]
\Function{ChooseElevator}{$call, elevatorSet, costFunction$}
  \State $selectedElevator \leftarrow null$
  \State $minCost \leftarrow \infty$
  \For{\textbf{each} $elevator$ in $elevatorSet$}
    \State $cost \leftarrow \Call{CostFunction}{$call, elevatorSet$}$
    \If{$cost = 0$}
      \State \textbf{return} $elevator$
    \EndIf
    \If{$cost < minCost$}
      \State $selectedElevator \leftarrow elevator$
      \State $minCost \leftarrow cost$
    \EndIf
  \EndFor
  \State \textbf{return} $selectedElevator$
\EndFunction
\end{algorithmic}
\end{center}
\caption
   {\label{alg:base}Algoritmo Base}
\end{algorithm}

Os argumentos para este algoritmo são:

\begin{description}[leftmargin=!,labelwidth=\widthof{\bfseries $costFunction$}]
  \item[$call$] Abstração de uma chamada;
  \item[$elevatorset$] Conjunto de todos os elevadores do prédio;
  \item[$costFunction$] Ponteiro para a função de cálculo de custo.
\end{description}

Seu funcionamento pode ser descrito como: dada uma chamada, para cada elevador
presente no conjunto calcula-se o retorno da função de custo e o algoritmo
retorna o elevador com a menor função de custo correspondente. Além disso, há
uma otimização (programação dinâmica): se algum elevador possuir custo zero, o
algoritmo retorna imediatamente, como resultado, este elevador - pois nenhum
outro elevador poderá ter um retorno da função de custo melhor do que zero.

\subsection{\label{sec:ai:nn}Função de Custo \textit{Nearest Neighbour}}

A função de custo \textit{Nearest Neighbour}, ou \textit{Vizinho Mais Próximo},
é a mais simples de todas e servirá como base para comparação e avaliação dos
demais algoritmos e funções. Seu funcionamento é trivial: na ocorrência de uma
chamada, a mesma será atendida pelo elevador que encontra-se mais próximo do
andar de origem, independente de quaisquer outras variáveis. Ou seja, o custo
para atender uma chamada é a diferença do andar em que o elevador se encontra e
do andar que originou a chamada, conforme mostra o algoritmo
\ref{alg:nn}.

\begin{algorithm}[htb]
\begin{center}
\begin{algorithmic}[1]
\Function{NearestNeighbour}{$call, elevatorSet$}
  \State \textbf{return} \Call{abs}{$elevator.currentFloor - call.sourceFloor$}
\EndFunction
\Statex
\Function{Abs}{value}
  \If{$value > 0$}
    \State \textbf{return} $value$
  \EndIf
  \State \textbf{return} $-1 * value$
\EndFunction
\end{algorithmic}
\end{center}
\caption
   {\label{alg:nn}Nearest Neighbor}
\end{algorithm}

Um dos muitos problemas deste algoritmo é que ele pode causar muitas mudanças de
direção de um elevador, o que acarreta num tempo de espera maior para os
passageiros dele. O único propósito deste algoritmo é servir de base de
comparação com outros algoritmos propostos, de modo a validarmos o simulador.
Espera-se que uma melhora clara de desempenho seja notado ao comparar-se este
com o próximo dos mais triviais, o \textit{Nearest Neighbour Melhorado}.

% TODO: Quem fala nele?

\subsection{\label{sec:ai:nnm}Função de Custo \textit{Nearest Neighbour
Melhorado}}

Uma melhoria que pode ser feita ao algoritmo de \textit{Nearest Neighbour}
é considerar o sentido em que o elevador está indo para atender o chamado. Isto
implica em considerar-se agora a informação de sentido dos pedidos. É importante
notar que ainda não se considera quantas pessoas fizeram um pedido para qual
sentido~-~apenas sabe-se que há pedidos no andar, e como destinos tem-se ``para
cima'', ``para baixo'' ou ambos.

\begin{algorithm}[htb]
\begin{center}
\begin{algorithmic}[1]
\Function{BetterNearestNeighbour}{$call, elevatorSet$}
  \State \textbf{return} $???????$
\EndFunction
\end{algorithmic}
\end{center}
\caption
   {\label{alg:nnm}Nearest Neighbor Melhorado}
\end{algorithm}

Este algoritmo resolve o problema de mudanças de direção que o algoritmo de
\textit{Nearest Neighbour} sofre.

No entanto, sua escolha para este trabalho também se dá para fim de comparação
com os outros e validação do simulador. Como seu comportamento é diferente do
caso mais trivial, mas ainda assim bastante simples, poderá ser visto com clareza
algum tipo de melhora no tempo de resposta do sistema simulado, bem como a validade do simulador.

\section{\label{sec:ai:lotacao}Função de Custo \textit{Lotação-Deslocamento}}

O primeiro algoritmo de IA a ser testado é simples: define-se uma
função de custo, inerente a cada elevador, que descreve matematicamente quão
vantajoso é atender um pedido, comparado a não atendê-lo. A decisão de qual
elevador é escolhido para atender o pedido é feita com base única e
exclusivamente em qual deles terá o menor valor da função de custo.

Um exemplo de função de custo é:

\[
  J(e) = g_{e}f_{e}
\]

Onde:
\begin{itemize}
\item \textbf{$g_{e}$} é a lotação do elevador
\item \textbf{$f_{e}$} é
  \begin{itemize}
    \item zero, caso o programa atual do elevador faça com que ele passe por
      aquele andar
    \item o número de andares entre onde o elevador está e o pedido, caso o
      elevador não tenha um programa (\textit{i.e.}, ele esteja ocioso) ou o
      elevador esteja indo na direção do pedido.
    \item duas vezes o número de andares entre onde o elevador está e o pedido,
      caso ele mude de direção para atender este pedido.\footnote{Multiplica-se
        a distância por dois pois é necessário ir até o andar do pedido e então
        voltar para o andar onde se estava anteriormente, para só então atender
        o pedido.}
  \end{itemize}
\end{itemize}

\begin{figure}[htb!]
  \centering
  \includegraphics[scale=0.6]{img/elevator_example1.eps}
  \caption{Exemplo de sistema de 3 elevadores com 8 andares, com um pedido para
    descer a partir do oitavo andar. A lotação dos elevadores está representada
    em percentual dentro deles, e seus programas são representados por setas.}
  \label{fig:elevadores-1}
\end{figure}

Utilizando o exemplo da Figura~\ref{fig:elevadores-1}, podemos calcular o custo
de cada elevador. Nela, temos 3 elevadores:

\begin{itemize}
\item \textbf{$E_{1}$}, no sétimo andar, com carga $20\%$ e um destino: o segundo andar;
\item \textbf{$E_{2}$}, no primeiro andar, com carga $10\%$ e um destino: o sexto andar;
\item \textbf{$E_{3}$}, no sétimo andar, com carga $90\%$ e um destino: o primeiro andar;
\end{itemize}

Sabemos que há um pedido no oitavo andar, e que ele quer descer\footnote{Na
  Figura~\ref{fig:elevadores-1}, representado pelo círculo com um \textbf{D}, de
\textit{Down}.}.

Para o Elevador $E_{1}$:

\[J(1) = g_{1}f_{1} = 0.2 \times 2 = 0.4\]

$g_{1}$ é $0.2$, pois sua lotação é $20\%$, e $f_{1}$ é $2$, pois o elevador
$E_{1}$ deve subir do sétimo para o oitavo andar e descer novamente até o sétimo.

Para o Elevador $E_{2}$:

\[J(2) = g_{2}f_{2} = 0.1 \times 8 = 0.8\]

$g_{2}$ é $0.1$, pois sua lotação é $10\%$\footnote{Como não temos informações a
respeito do futuro, não podemos considerar alterações na carga de um elevador.},
e $f_{2}$ é $8$, pois o elevador deverá subir oito andares.

Para o Elevador $E_{3}$:

\[J(3) = g_{3}f_{3} = 0.9 \times 2 = 1.8\]

$g_{3}$ é $0.9$, pois sua lotação é $90\%$, e $f_{3}$ é $2$, pois o elevador
$E_{3}$ deve subir do sétimo para o oitavo andar e descer novamente até o sétimo.

Vemos que, para esta função de custo, neste sistema, é vantajoso mudar o sentido
de $E_{1}$ para atender o chamado no oitavo andar.

Várias funções de custo podem ser experimentadas e comparadas.

Outras funções de custo levariam em consideração mudanças de direção de viagem
\footnote{\textit{e.g.}, pode ser vantajoso um elevador mudar de direção para atender um
pedido a um andar de distância, caso a alternativa seja fazer o pedido esperar
um deslocamendo de dezenas de andares de outro elevador.}, ou ainda tentar
manter todos os custos o mais baixo possível, ao mesmo tempo que sejam todos o
mais próximos uns dos outros.

\subsection{\label{sec:ai:lotacaoquadratica}Função de Custo
\textit{Lotação-Deslocamento Quadrático}}

Uma outra função de custo possível seria:

\[J(e) = \sqrt{g_{e}f_{e}^{2}}\]

Esta função penaliza mais o movimento, elevando ele ao quadrado.

Para o exemplo da Figura~\ref{fig:elevadores-1}, temos:

\[J(1) = \sqrt{g_{1}f_{1}^{2}} = \sqrt{0.2 \times 2^2} = \sqrt{0.8} = 0.8944\]
\[J(2) = \sqrt{g_{2}f_{2}^{2}} = \sqrt{0.1 \times 8^2} = \sqrt{6.4} = 2.5298\]
\[J(3) = \sqrt{g_{3}f_{3}^{2}} = \sqrt{0.9 \times 2^2} = \sqrt{3.6} = 1.8973\]

Novamente, para esta função, o Elevador $E_{1}$ é eleito para atender o pedido.

\section{Planning}

% TODO: essa explicação pode melhorar ainda

A idéia do algoritmo de planning é estender o de função de custo, calculando a
mesma para vários passos no futuro~-~como em um jogo de xadrez. Busca-se o
caminho cuja soma tem o menor custo.

Há duas alternativas para esta árvore de decisões: a decisão pode ser ``qual
elevador deve atender o próximo pedido'' (Figura~\ref{fig:planning}),
ou ``qual pedido deve ser atendido por qual elevador''. % TODO: Esta ilustração

O horizonte de cálculo deve ser selecionado, dado que é um fator limitante no
tamanho do cálculo do algoritmo.

Cada decisão diferente, para cada elevador, é um nodo novo na árvore. Ao
escolher-se uma das alternativas (\textit{i.e.} a que, ao final de $x$ eventos
no futuro tem o menor custo), avança-se um passo na simulação e executa-se o
algoritmo novamente.

Na Figura~\ref{fig:planning}, vemos um exemplo de planning sendo executado com
horizonte 3 e dois elevadores. Para cada passo do algoritmo, a decisão a ser
tomada é ``qual elevador deve atender o próximo pedido da fila?'', e, para isto,
calcula-se a função de custo. O resultado da função pode ser visto entre
parênteses em cada nodo da árvore da Figura~\ref{fig:planning}.

Ao encontrarmos o horizonte (no caso da Figura~\ref{fig:planning}, no nível 3 da
árvore), soma-se os custos até lá (na Figura~\ref{fig:planning}, os círculos
abaixo do nível mais baixo indicam a soma dos custos). O caminho que leva ao
menor custo é o que deve ser tomado. No exemplo da Figura~\ref{fig:planning}, o
elevador $E2$ atenderá o primeiro e o segundo chamados da fila, e então o
elevador $E1$ atenderá o terceiro chamado, como pode ser visto pelo caminho destacado.

\begin{figure}[htb!]
  \centering
  \includegraphics[scale=0.6]{img/planning.eps}
  \caption{Exemplo de planning com horizonte 3 e dois elevadores}
\label{fig:planning}
\end{figure}

\section{Planning Multi-Agente}

Este algoritmo é uma extensão do algoritmo de Planning, onde, em vez de termos
um processamento central que decide que um elevador deve atender o pedido, temos
todos os elevadores calculando por conta própria se vale a pena atender um
pedido ou não~-~sem conhecimento do estado dos demais.

A literatura neste tópico é bastante escassa ainda.
\chapter{\label{chap:simulation}Sistemas e Simulações}

De acordo com Banks~(2005, p.~3):

\begin{directcite}
Simulação é a imitação da operação de um sistema ou processo do mundo real ao
longo do tempo. Ela envolve a geração de uma história artificial de um sistema
ou processo e a observação desta história de modo a realizar inferências a
respeito das características operacionais da realidade ali representada.
\end{directcite}

Entretanto, a simulação não é a única abordagem existente para se estudar e
compreender um sistema e suas características. É senso comum que cada sistema,
possuindo suas próprias características e idiossincrasias, deve ser analisado
através da ferramenta correta. Logo, embora a simulação pareça ser uma boa
alternativa à primeira vista em diversos casos, é possível que ela não seja a
forma mais apropriada para o seu estudo. Assim, se faz importante a existência
de métodos objetivos para que se possa verificar se a simulação é realmente a
ferramenta apropriada para cada caso estudado.

\section{\label{chap:waystostudy}Formas de Estudar um Sistema}

Ao realizar o estudo sobre um sistema e seu comportamento, uma gama de
estratégias podem ser colocadas em prática, dentre as quais podemos citar:

\begin{itemize}
\item Experimentar com o próprio sistema;
\item Experimentar com um modelo físico do sistema;
\item Experimentar com um modelo matemático do sistema.
\end{itemize}

A fim de decidir de forma objetiva qual a melhor abordagem para um dado sistema,
Law~\cite{Law} propõe uma reflexão (figura~\ref{fig:systemstudy}) através das
seguintes perguntas:

\begin{figure}[htb!]
\centering\includegraphics{img/systemstudy.eps}
\caption{\label{fig:systemstudy}Formas de estudar um sistema. Fonte:~\cite{Law}}
\end{figure}

\begin{enumerate}
\item \textit{Experimentar com o próprio sistema ou experimentar com um modelo do mesmo?}

Caso exista viabilidade técnica e financeira na alteração de um sistema e na
observação de sua operação sob estas novas condições, é desejável que se faça as
experimentações diretamente no próprio sistema. Entretanto, frequentemente esta
alternativa é inviável por diversas razões, dentre as quais encontram-se:

\begin{itemize}
  \item O custo dos experimentos é muito elevado:

  \textbf{Exemplo}: construir uma instalação para um novo elevador em um prédio
já existente.

  \item O impacto causado pelos experimentos pode ser prejudicial ao sistema:

  \textbf{Exemplo}: um mercado deseja reduzir o número de atendentes em caixas a
fim de diminuir seus custos operacionais e aumentar seus lucros. Porém, esta
medida pode causar um aumento significativo no tamanho das filas - e,
consequentemente, um aumento no tempo de espera dos cliente, podendo até causar
desistências.

  \item O sistema não existe:

  \textbf{Exemplo}: o sistema ainda está nas fases de concepção, projeto ou
desenvolvimento.

\end{itemize}

Por isto, muitas vezes é necessário realizar os experimentos utilizando um
\textit{modelo} em substituição ao sistema real. No escopo desde estudo, um
sistema de elevadores operacional e pronto para realizar as experimentações não
encontra-se entre os recursos disponíveis para a elaboração da pesquisa. Ainda,
mesmo que tal estrutura estivesse disponível, os cenários de testes possíveis
seriam limitados pelas restrições físicas do sistema, tornando o estudo das
melhorias mais caro e menos flexível. Logo, optou-se pela utilização de um
\textit{modelo do sistema}.

\item \textit{Experimentar utilizando um modelo físico do sistema ou utilizar um modelo
matemático?}

Cockpits de aviões desacoplados do resto da nave, automóveis construídos em
escala em túneis de vento e miniaturas funcionais de sistemas de elevadores são
alguns exemplos de \textit{modelos físicos} de sistemas. Ocasionalmente, este
tipo de modelo é utilizado para estudar sistemas de engenharia ou
logística~\cite{Law}. Porém, a vasta maioria dos casos exigem \textit{modelos
matemáticos}. Estes modelos são criados baseados em suposições realizadas a
respeito do funcionamento do sistema. Tais suposições, que normalmente possuem a
forma de relações lógicas e quantitativas, são então avaliadas numericamente
enquanto são coletados dados com o intuito de observar de que forma o sistema
real reagiria.

No contexto deste estudo, um modelo físico de um sistema de elevadores poderia
ser constituído por uma maquete de um prédio com mini-elevadores movidos à
motores de passo, que por sua vez seriam controlados por microcontroladores
programáveis conectados à uma rede de sensores. Este projeto por si só,
entretanto, já seria grandioso demais - além de, obviamente, fugir do escopo da
Ciência da Computação e ser mais adequado à um trabalho de conclusão de
Engenharia Elétrica ou Engenharia de Controle e Automação. Além disso, da mesma
forma que em um sistema real de elevadores, os cenários de testes possíveis
seriam limitados pelas restrições físicas do modelo do sistema. Portanto, optou-
se pela utilização de um \textit{modelo matemático do sistema}, reproduzível em
ambiente computacional e parametrizável para diferentes cenários.

\item \textit{O problema pode ser resolvido de forma analítica?}

De posse de um \textit{modelo matemático de um sistema}, o mesmo deve ser
examinado de modo a verificar de qual modo ele pode ser utilizado para dar
solução ao problema que ele representa. Segundo~\cite{Law}, se o modelo for
simples o suficiente, provavelmente é possível trabalhar com suas relações
matemáticas e equações para obter uma solução exata, anaĺitica. \cite{Law} cita
como exemplo o modelo de mecânica básica $d = vt$, onde $d$ é a distância, $v$ é
a velocidade média e $t$ é o tempo de viagem. Se soubermos a distância a ser
viajada e a velocidade, podemos trabalhar com as equações do modelo e encontrar
a equação $t = d/r$ para encontrar, exatamente, o tempo de viagem. Embora esta
solução seja simples, não é incomum a busca por soluções analíticas tornar-se
extraordinariamente complexa, exigindo vastos recursos computacionais.

Portanto, se uma solução analítica para um problema é conhecida e possui
eficiência computacional, é mais apropriado estudar este sistema desta forma do
que através de simulação. Do contrário, o sistema deve ser estudado através de
simulações~-~ou~seja, através do exercício numérico das entradas do modelo
matemático do sistema e da observação da forma com que tais exercícios afetam as
saídas. Conforme afirmado no capítulo \ref{chap:intro} deste estudo, o problema
de atribuir elevadores para atender chamadas feitas pelos passageiros
minimizando alguma métrica encontra-se no conjunto de problemas NP- difícil (ou
NP-hard, ou NP-complexo)~\cite{SeKo99}. Assim, uma solução ótima, computável em
tempo polinomial, ainda não é conhecida para este problema. Este fato vai ao
encontro dos grandes esforços da indústria em procurar soluções para resolver o
problema ao longo das décadas, não poupando esforços e investimentos em busca
desta solução.

\end{enumerate}

Em outra abordagem, Banks~\cite{BanksGibson} define 10 situações onde a simulação
\textbf{não} é indicada. Tais situações são:

\begin{enumerate}
\item \textit{Quando o problema puder ser resolvido utilizando-se de bom senso}

Conforme discutido anteriormente neste capítulo, dada a complexidade do problema
(NP-completo), não parece ser possível encontrar a solução apenas utilizando o
bom senso e operações básicas.

\item \textit{Quando o problema puder ser resolvido de forma analítica}

Sobre o problema em questão, uma solução anaĺítica ainda é desconhecida.

\item \textit{Quando o problema puder ser resolvido através de experimentos
diretamente no sistema}

Como já constatado, não há um sistema de elevadores instalado para realizar as
experimentações entre os recursos disponíveis para este estudo. E, mesmo que
houvesse, haveria uma imposição de limites nos cenários de testes em função das
restrições físicas do sistema. Logo, não é possível realizar os experimentos
diretamente no sistema.

\item \textit{Quando seus custos da simulação excederem os seus ganhos}

Não foi realizado um estudo e avaliação a respeito dos custos deste estudo,
compreendendo o embasamento, modelagem, projeto e implementação do simulador.
Entretanto, acredita-se que os resultados trarão benefícios aos usuários de
elevadores, conforme defendido nos capítulos \ref{chap:intro},
\ref{chap:problem} e \ref{chap:proposal}.

\item \textit{Quando não há recursos financeiros suficientes}

Idem ao item anterior.

\item \textit{Quando não há tempo suficiente}

De acordo com o cronograma exibido na seção \ref{chap:stages}, acredita-se que
será possível implementar um sistema simulador capaz de fornecer dados em tempo
hábil.

\item \textit{Quando não há dados suficientes, nem mesmo estimativas}

Neste estudo serão utilizadas distribuições de probabilidade baseadas em
estimativas obtidas através de observação e senso comum.

\textbf{TO-DO: Onde observaremos?} % TO-DO

\item \textit{Quando não há possibilidade de verificar a validade do modelo}

\textbf{TO-DO: Como verificaremos a validade do modelo?} % TO-DO

\item \textit{Quando as expectativas e o poder da simulação são superestimados}

Neste estudo, o foco das expectativas está nos ganhos que o uso de algoritmos de
Inteligência Artificial poderão trazer para estes sistemas de elevadores. A
simulação é vista apenas como uma forma para avaliar os resultados obtidos.
Portanto, ela não é superestimada e o objeto de estudo não se enquadra nesta
situação.

\item \textit{Quando o comportamento do sistema é muito complexo ou não pode ser
definido (e.~g. seres humanos)}

Um \textbf{EGCS} não é um sistema de alta complexidade, possuindo um conjunto de
comportamentos limitado e um espaço de estados moderado. Portanto, não se
enquadra nesta situação.

\end{enumerate}

Em função destas reflexões acredita-se que a simulação é uma forma apropriada
para estudar um \textbf{EGCS}.

\section{Classificação do Modelo de Simulação}

A partir de um modelo matemático a ser estudado por meio de simulação (doravante
chamado de \textit{modelo de simulação}), o mesmo pode ser classificado em três
dimensões~\cite{Banks,Law}:

\begin{enumerate}
\item \textit{Estático ou Dinâmico}

Um modelo de simulação estático é uma representação de um sistema em um ponto
particular no tempo, ou então um sistema onde o tempo simplesmente é
irrelevante. Por outro lado, um modelo de simulação dinâmico representa um
sistema que evolui e se modifica com o passar do tempo.

No escopo deste trabalho, o modelo de simulação é claramento dinâmico. Deseja-se
verificar o comportamento do sistema ao longo de um intervalo de tempo finito, à
medida que passageiros chegam e elevadores os transportam através dos andares do
prédio.

\item \textit{Determinístico ou Estocástico}

Um modelo de simulação que não possua nenhum componente probabilístico (i.~e,~
aleatoriedade) é chamado de \textit{determinístico}. Em um modelo de simulação
\textit{determinístico} sua saída é ``determinada'' no momento em que a sua
entrada é definida - mesmo que ainda tome tempo computacional para realizar o
cálculo de qual seja o resultado \cite{Law}. Em outras palavras, significa dizer
que a saída fornecida pelo modelo de simulação será sempre a mesma para uma
mesma entrada. Porém a grande maioria dos sistemas do mundo real possuem, no
mínimo, algum grau de aleatoridade na sua entrada. Estes sistemas dão origem a
modelos de simulação \textit{estocásticos}. Estes, diferentemente de modelos
\textit{determinísticos}, fornecem uma saída igualmente aleatória~-~e, por esta
razão, esta saída deve ser considerada como um conjunto de \textit{estimativas}
das características reais do sistema, e não as características propriamente
ditas \cite{Banks}.

Em um \textbf{EGCS} real não é possível prever quantos passageiros utilizarão o
sistema, quando eles chegarão e tampouco para qual andar irão. Portanto, um
modelo de simulação válido neste escopo deve ser \textit{estocástico} de modo a
lidar com a aleatoridade da entrada de passageiros no sistema.

\item \textit{Contínuo ou Discreto}

A figura \ref{fig:disccont} ilustra o comportamento de uma variável de estado em
modelos de simulação \textit{contínuos} e \textit{discretos}. O modelo
\textit{contínuo} é aquele onde os valores das variáveis mudam continuamente ao
longo do tempo. Já em um modelo \textit{discreto} as variáveis de estado tem seu
valor alterado em instantes separados do tempo \cite{Banks}. É importante
salientar que modelos discretos não necessariamente representam sistemas
discretos e vice-versa \cite{Law}. A decisão por um modelo ou outro se dá
especificamente pelos objetivos do estudo.

Para este projeto, não há a necessidade - tampouco a disposição de tempo
computacional - de informações instantâneas a respeito da movimentação de
passageiros e elevadores, e sim de observar o comportamento do sistema dada a
ocorrência de determinados eventos, e.~g., um passageiro embarca em um elevador.
Portanto, no escopo deste estudo utilizaremos um modelo de simulação
\textit{discreto}.

\begin{figure}[htb!]
\centering\includegraphics{img/discrete_continuous.eps}
\caption{\label{fig:disccont}Variável de estado em um modelo contínuo (A) e discreto (B). Fonte:~\cite{Banks}}
\end{figure}

\end{enumerate}

\section{Simulação Baseada em Eventos Discretos}

Segundo Law~(2000, p.~6):

\begin{directcite}
A Simulação de Eventos Discretos (\textit{Discrete-Event Simulation}) compreende
a modelagem de um sistema à medida que ele \textbf{evolui ao longo do tempo}
através de uma representação na qual as variáveis de estado são alteradas
instantaneamente em \textbf{instantes separados no tempo}.
\end{directcite}

A abordagem sugerida por Law é o padrão de projeto de simuladores ao utilizar-se
um modelo de simulação \textit{dinâmico}, \textit{estocástico} e
\textit{discreto} para representar o sistema do mundo real. Em função da
natureza dinâmica desta abordagem, é necessário acompanhar o valor atual do
tempo da simulação à medida que a simulação é executada, armazenando este valor
em uma variável. Esta variável é chamada de \textit{relógio da simulação}
\cite{Law}. Também se faz necessário um mecanismo de avanço de tempo que
gerencie o valor desta variável. Geralmente, não há relação entre o tempo de
simulação e o tempo necessário para a simulação ser executada. Por exemplo, um
experimento pode simular o funcionamento de um banco entre 9h e 17h (tempo de
simulação), mas o tempo necessário para executar a simulação poderia ser de 4
minutos.

\subsection{Mecanismo de Avanço de Tempo}

Existem duas principais abordagens para o mecanismo de avanço de tempo em um sistema de simulação de eventos discretos. São eles:

\begin{description}
\item[Avanço de tempo para o próximo evento] \hfill

Na abordagem de \textit{avanço de tempo para o próximo evento}, o
\textit{relógio da simulação} é inicializado em 0 e é determinado em que ponto
tempo ocorrerão eventos futuros - em outras palavras, é feito o agendamento de
eventos. Então, o \textit{relógio da simulação} é avançado para o tempo da
ocorrência do \textit{primeiro} destes eventos futuros. Neste ponto do tempo, o
estado do modelo é atualizado de acordo com o evento que ocorreu e novas
ocorrências de eventos futuros são agendadas. Então, o \textit{relógio da
simulação} avança para o instante exato da ocorrência do \textit{novo} primeiro
dos eventos futuros e o estado do sistema é atualizado em função da ocorrência
deste evento. Este processo repete-se até que uma condição de parada previamente
definida seja satisfeita.

Uma vez que todas e qualquer alteração no estado do sistema ocorre na ocasião de
um evento, os períodos de espera, que são o tempo entre a ocorrência de um
evento e o próximo - não são relevantes para a simulação. Afinal, o estado do
sistema não foi alterado neste ínterim - ou seja, nada de interessante ocorreu
durante aquele tempo \cite{Sim}. Deste modo, é possível reduzir o esforço
computacional necessário para executar a simulação.

\begin{figure}[htb!]
\centering\includegraphics{img/nextevent.eps}
\caption{\label{fig:nextevent}Ilustração da evolução do \textit{relógio da simulação} utilizando a abordagem de \textit{avanço de tempo para o próximo evento}. Fonte:~\cite{Law}}
\end{figure}

Um exemplo desta dinâmica é ilustrado pela figura \ref{fig:nextevent}. O eixo do
tempo, iniciado em 0, marca os tempos agendados para os eventos $\{e_{0}, e_{1},
e_{2}, e_{3}, e_{4}, e_{5}\}$. As setas indicam os valores assumidos pelo
\textit{relógio da simulação}, ou seja, $\{t_{0}, t_{1}, t_{2}, t_{3}, t_{4},
t_{5}\}$. Observa-se que $t_{0} = e_{0}$, $t_{1} = e_{1}$ e assim sucessivamente
- ou seja, o \textit{relógio da simulação} avança diretamente para o momento
da ocorrência do próximo evento.

\item[Avanço de tempo de incremento fixo] \hfill

Na abordagem de \textit{avanço de tempo de incremento fixo}, o \textit{relógio
da simulação} é alterado sempre no mesmo incremento, independente da ocorrência
de eventos. A cada avanço é feita uma varredura para verificar se haviam eventos
agendados dentro daquele intervalo. Se um ou mais eventos estavam agendados para
ocorrer no decorrer do intervalo, considera-se que sua ocorrência se deu no
\textit{fim} do intervalo e o estado do sistema é alterado de forma
correspondente.

\begin{figure}[htb!]
\centering\includegraphics{img/fixed.eps}
\caption{\label{fig:fixedtime}Ilustração da evolução do \textit{relógio da simulação} utilizando a abordagem de \textit{avanço de tempo de incremento fixo}. Fonte:~\cite{Law}}
\end{figure}

A representação ilustrada pela figura \ref{fig:fixedtime} mostra a diferença
deste mecanismo para o anterior. O eixo do tempo, iniciado em 0, marca os tempos
agendados para os eventos $\{e_{1}, e_{2}, e_{3}, e_{4}\}$. As
setas indicam os valores assumidos pelo \textit{relógio da simulação}. Observa-
se que, diferentemente da abordagem anterior, o valor do \textit{relógio da
simulação} não coincide com a ocorrência de qualquer evento, e sim valores
múltiplos do incremento fixo $\Delta t$.

Algumas desvantagens desta abordagem são os erros introduzidos ao processar
eventos no fim do intervalo em que eles ocorreram e a necessidade de decidir
qual evento processar primeiro quando eventos que não ocorrem simultaneamente na
realidade são tratados como tal pelo modelo. Tais problemas poderiam ser
mitigados ao utilizar incrementos menores. Entretanto, esta medida também
aumenta o número de varreduras em intervalos para localizar eventos, o que pode
tornar sua execução muito mais custosa em termos computacionais.

Em função disto, esta abordagem é frequentemente descartada em detrimento do
\textit{avanço de tempo para o próximo evento} em modelos onde o tempo de
ocorrência de eventos pode variar bastante - que é o caso de modelos
estocásticos.

\end{description}

Sendo assim, o modelo deste estudo será uma simulação de eventos discretos com avanço de tempo para o próximo evento.

\subsection{Componentes e Organização da Simulação}

Independente do paradigma utilizado em sua implementação, um simulador de
eventos discretos possui um conjunto de componentes com responsabilidaddes bem
definidas. São eles:

\begin{description}
\item[Estado do Sistema] Coleção de variáveis necessárias para descrever o estado do sistema em um instante em particular;
\item[Relógio de Simulação] Variável contendo o valor atual do tempo de simulação;
\item[Lista de Eventos] Lista de eventos futuros, contendo, para cada evento, o
seu tipo e o instante do tempo no qual este ocorrerá;
\item[Contadores Estatísticos] Conjunto de variáveis utilizados para armazenar informações estatísticas a respeito do funcionamento do sistema;
\item[Rotina de Inicialização] Subprograma para inicializar a simulação no
instante zero do \textit{relógio da simulação};
\item[Rotina de Temporização] Subprograma responsável por determinar qual o
próximo evento a ocorrer e avançar o relógio de simulação até o instante da sua
ocorrência;
\item[Rotina de Evento] Subprograma que modifica o estado do sistema quando um determinado tipo de evento ocorre - para cada tipo de evento há uma rotina
correspondente;
\item[Rotinas Auxiliares] Conjunto de subprogramas utilizados para gerar observações aleatórias sobre distribuições de probabilidade determinadas como parte do modelo de simulação;
\item[Gerador de Relatórios] Subprograma que computa estimativas das métricas desejadas e dão origem ao relatório quando a simulação termina;
\item[Programa Principal] Subprograma que invoca as rotinas de inicialização e temporização e delega o tratamento de cada evento à sua rotina correspondente, repetindo o processo até que a condição de parada da simulação seja verificada.
\end{description}

\begin{figure}[htb!]
\centering\includegraphics{img/simulation_flow.eps}
\caption{\label{fig:simflow}Fluxo básico de um simulador com avanço de tempo para o próximo evento. Fonte:~\cite{Law}}
\end{figure}

A figura \ref{fig:simflow} ilustra a organização existente entre estes
componentes, bem como o fluxo de execução da simulação. O início da simulação se
dá pela execução do \textit{Programa Principal}. Este, por sua vez, invoca a
\textit{Rotina de Inicialização}. Nesta rotina, o \textit{Relógio da Simulação}
é zerado, marcando o instante inicial. Além disso, são inicializados o
\textit{Estado Inicial} do sistema, os \textit{Contadores Estatísticos} e são
gerados os eventos iniciais, sendo adicionados à \textit{Lista de Eventos}.

Após o controle retornar para o \textit{Programa Principal}, o mesmo invoca a
\textit{Rotina de Temporização}. Esta rotina tem a responsabilidade de avaliar a
\textit{Lista de Eventos} e determinar qual é o primeiro evento a ser executado.
Ao determinar qual é o evento, a rotina avança o \textit{Relógio da Simulação}
para o instante de sua ocorrência e devolve informações a respeito deste evento
para o \textit{Programa Principal}. De volta ao \textit{Programa Principal}, o
mesmo avalia o evento retornado pela \textit{Rotina de Temporização} e invoca
sua \textit{Rotina de Evento} correspondente.

Para cada tipo de evento existe uma \textit{Rotina de Evento} correspondente.
Todas elas, obrigatoriamente, devem executar as seguintes tarefas: (1) atualizar
o \textit{Estado do Sistema} em função do evento ocorrido; (2) atualizar os
\textit{Contadores Estatísticos} em função do evento ocorrido; e (3) gerar
novos eventos futuros e adicioná-los à \textit{Lista de Eventos}. Para esta
última tarefa, as \textit{Rotinas Auxiliares} são utilizadas para diversas
tarefas - dentre elas, a geração de variáveis aleatórias para dar suporte à
geração de eventos estocásticos.

Após a execução da \textit{Rotina de Evento}, é feita a verificação de uma
condição de parada da simulação. Caso seja determinado que o fim da simulação
deve ocorrer, o controle passa para o \textit{Gerador de Relatórios}, cuja
responsabilidade é computar as estimativas de interesse, baseando-se nos
\textit{Contadores Estatísticos} e produzir um relatório com estes dados. Em
caso contrário, o \textit{Programa Principal} é invocado novamente e o ciclo se
repete. É importante salientar que a \textit{Rotina de Inicialização} somente é
invocada na primeira execução do \textit{Program Principal}, sendo ignorada nas
iterações posteriores.

%\begin{algorithm}[htb]
%\begin{center}
%\begin{algorithmic}[1]
%\Function{Initialize}{$clock, state, statistics$}
%  \State $clock \gets 0.0$
%  \State Initialize $state$
%  \State Initialize $statistics$
%\EndFunction
%\Statex
%\Function{Timing}{$clock, state, statistics$}
%  \State $clock \gets 0.0$
%  \State Initialize $state$
%  \State Initialize $statistics$
%\EndFunction
%\Statex
%\Function{Main}{}
%  \State $clock \gets null$
%  \State $state \gets null$
%  \State $statistics \gets null$
%  \State $eventList \gets \text(new queue<Event>)$
%  \State \Call{InitializationRoutine}{clock, state, statistics}
%  \While{$\Call{ShouldSimulationRun}{clock, state}$}
%    \State $nextEvent \gets \Call{GetNextEvent}{}$
%    \If{<text>}
%    <body>
%    \ElsIf{<text>}
%    <body>
%    \Else
%    <body>
%    \EndIf
%  \EndWhile
%\EndFunction
%\Procedure{Euclid}{$a,b$}\Comment{The g.c.d. of a and b}
%\State $r\gets a\bmod b$
%\While{$r\not=0$}\Comment{We have the answer if r is 0}
%\State $a\gets b$
%\State $b\gets r$
%\State $r\gets a\bmod b$
%\EndWhile\label{euclidendwhile}
%\State \textbf{return} $b$\Comment{The gcd is b}
%\EndProcedure
%\end{algorithmic}
%\end{center}
%\caption[An algorithm with an optional, shorter caption]%
%    {\label{alg:alg1}This is an algorithm with a very long
%    caption. However, we replaced it with a shorter version
%    in the Outline for legibility reasons}%
%\end{algorithm}
\include{chap-modeling}
\include{chap-related}
\include{chap-conclusion}

\bibliographystyle{tcc-num}
\bibliography{bib-proposta}

\end{document}
